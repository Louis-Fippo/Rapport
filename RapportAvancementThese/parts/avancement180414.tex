% Diapo des avancements

\begin{frame}[c]
  \frametitle{Avancements}
 
  \fbox{\tval{\large Test sur les 8 autres gènes}} 
 
%\pause
\begin{itemize}
  \item Test très difficil, 
  \item C'est irréaliste de donner une valeur commune pour l'auto frappe
  \item Définir une méthode commune qui sera fonction de l'expression de chaque gènes
  \item Il y aura probablement des pertes d'informations chez certains gènes(à moins d'ajouter des processus au niveau des FT)
\end{itemize}

\fbox{\tval{\large Méthodes moins arbitraire}} 
\begin{itemize}
 \item Pour les autos frappes: la fréquence de l'auto frappe doit être estimé pour chaque gène avec biensûr le même principe( à expliquer).
 \item 
\end{itemize}

\fbox{\tval{\large Plus généralement}} 
\begin{itemize}
 \item Remplacer les temps des frappes estimés par les fréquences.( à détailler)
\end{itemize}


%\textcolor{couleurtheme}{$\Rightarrow$} \fbox{\tval{\large Analyse???}} \textcolor{couleurtheme}{$\Leftarrow$}

\end{frame}


% ici on va faire un petit test pour Sweave
\begin{frame}[c]
\frametitle{Avancements}
 
  \fbox{\tval{\large Intégrer les sous graphes}} 
\begin{itemize}
 \item  Pour l'instant ça marche bien quand je met ensemble p114 et p115.
\end{itemize}

%illutration
\begin{center}
  \includegraphics[scale=0.2]{figs/sous-g-ens.jpg}
\end{center}

\end{frame}[c]

\begin{frame}[c]
\frametitle{Avancements}
 
  \fbox{\tval{\large Introduire les boucles}} 
\begin{itemize}
 \item  L'introduction de la boucle ne modifie pas l'expression de Ecad(p11)
 \item soit le signal n'est pas observé (ou il n'a pas le temps de revenir)
 \item soit 
\end{itemize}

%illustrations avec une figure 

\end{frame}[c]

\begin{frame}[c]
\frametitle{Avancements}
 
  \fbox{\tval{\large Les illustrations }} 

%illustrations avec une figure 

\end{frame}[c]



