% Diapo des résolutions

\begin{frame}[c]
  \frametitle{Résolutions}


%\pause
\begin{itemize}
  \item Faire les simulations sur des plus grandes durées de temps
  \item Voir dans quelle(s) mesure(s) modifier le simulateur stochastique(pour éviter les frappes régulières)
  \item Introduire la dégradation
  \item Ne considérer que les fronts montants
  \item Voir comment on peut se servir de l'analyse statique pour détecter et analyser les simulations
  \item Vérifier les oscillations par analyse statique 
\end{itemize}

%\textcolor{couleurtheme}{$\Rightarrow$} \fbox{\tval{\large Voyons ce qui a été réalisé...}} \textcolor{couleurtheme}{$\Leftarrow$}

\end{frame}

\begin{frame}[c]
 \frametitle{Résolutions à froid}
 
 \begin{itemize}
  \item Faire une analyse des sous graphes
  \pause
   \begin{itemize}
    \item définir un protocole de reconstruction des sous graphes(en ne considérant que les noeuds nécessaires)
    \item faire des analyses sur ces sous graphes
    \item reconstruire le grand réseau par augmentation progressive des sous graphes.
   \end{itemize}

     %\subitem définir un protocole de reconstruction des sous graphes
    % \subitem faire des analyses sur ces sous graphes
     %\subitem reconstruire le grand réseau par augmentation progressive des sous graphes.
  \item Considérer les régulateurs comme étant toujours actif
  \item Introduire des priorités aux actions du Process Hitting
     ( séparer les actions(celles des sortes coopératives par exemple))
 \end{itemize}

\end{frame}
