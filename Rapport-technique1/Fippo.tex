\documentclass[11pt,a4paper,twoside]{epig}
\usepackage[latin1]{inputenc}
\usepackage{times}
\usepackage{helvet}
%package pour les algo
\usepackage{algorithm}
\usepackage{algorithmic}
\usepackage{comment}

\pagestyle{empty}
\newcommand{\smallitem}{\vspace*{-2mm}\item}
\renewcommand{\baselinestretch}{0.95}
\marginparwidth 0pt 
\evensidemargin 2.2cm 
\oddsidemargin 1.8cm 
\topmargin 0cm 
\textwidth 12cm 
\textheight 19cm

\makeatletter
\renewcommand{\fnum@figure}{\textbf{\fontsize{10}{10}\selectfont\sffamily\figurename~\thefigure}}
\renewcommand{\fnum@table}{\textbf{\fontsize{10}{10}\selectfont\sffamily\tablename~\thetable}}
\makeatother


\newcommand{\auteur}[1]{
	\begin{center}
	{#1}
	\end{center}
}

\newcommand{\titre}[1]{
	\begin{center}
	{\large\bfseries \sffamily {#1}}
	\end{center}
}

\newcommand{\affiliation}[1]{
	\noindent
	{\small{#1}}
}


\usepackage{hyperref}
\usepackage{amsmath}  % Maths
\usepackage{amsfonts} % Maths
\usepackage{amssymb}  % Maths
\usepackage{stmaryrd} % Maths (crochets doubles)

\usepackage{url}     % Mise en forme + liens pour URLs
\usepackage{array}   % Tableaux évolués

\usepackage{prettyref}
\newrefformat{def}{Def.~\ref{#1}}
\newrefformat{fig}{Fig.~\ref{#1}}
\newrefformat{pro}{Property~\ref{#1}}
\newrefformat{pps}{Proposition~\ref{#1}}
\newrefformat{lem}{Lemma~\ref{#1}}
\newrefformat{thm}{Theorem~\ref{#1}}
\newrefformat{sec}{Sect.~\ref{#1}}
\newrefformat{ssec}{Subsect.~\ref{#1}}
\newrefformat{suppl}{Appendix~\ref{#1}}
\newrefformat{eq}{Eq.~\eqref{#1}}
\def\pref{\prettyref}

\usepackage{ntheorem}
\theoremheaderfont{\fontsize{10}{10}\sffamily\bfseries}
%\newtheorem*{example}{Example}{\itshape}{}
\newtheorem{example}{Example}{\itshape}{}
\newtheorem{definition}{Definition}{\itshape}{}

\usepackage{tikz}
\newdimen\pgfex
\newdimen\pgfem
\usetikzlibrary{arrows,shapes}
%\usetikzlibrary{shadows,scopes}
%\usetikzlibrary{positioning}
%\usetikzlibrary{matrix}
%\usetikzlibrary{decorations.text}
%\usetikzlibrary{decorations.pathmorphing}



%% Macros relatives à la traduction de PH avec arcs neutralisants vers PH à k-priorités fixes

% Macros générales
\newcommand{\ie}{\textit{i.e.} }

% Notations générales pour PH
\newcommand{\PH}{\mathcal{PH}}
\newcommand{\PHs}{\mathcal{S}}
%\newcommand{\PHp}{\mathcal{P}}
\newcommand{\PHp}{\textcolor{red}{\mathcal{P}}}
\newcommand{\PHproc}{\mathcal{P}}
\newcommand{\PHa}{\mathcal{A}}
\newcommand{\PHl}{\mathcal{L}}
\newcommand{\PHn}{\mathcal{N}}

\newcommand{\PHfrappeur}{\mathsf{frappeur}}
\newcommand{\PHcible}{\mathsf{cible}}
\newcommand{\PHbond}{\mathsf{bond}}
\newcommand{\PHsorte}{\mathsf{sorte}}
\newcommand{\PHbloquant}{\mathsf{bloquante}}
\newcommand{\PHbloque}{\mathsf{bloquee}}

\newcommand{\PHfrappeR}{\textcolor{red}{\rightarrow}}
\newcommand{\PHmonte}{\textcolor{red}{\Rsh}}

\newcommand{\PHfrappeA}{\rightarrow}
\newcommand{\PHfrappeB}{\Rsh}
%\newcommand{\PHfrappe}[3]{\mbox{$#1\PHfrappeA#2\PHfrappeB#3$}}
%\newcommand{\PHfrappebond}[2]{\mbox{$#1\PHfrappeB#2$}}
\newcommand{\PHfrappe}[3]{#1\PHfrappeA#2\PHfrappeB#3}
\newcommand{\PHfrappebond}[2]{#1\PHfrappeB#2}
\newcommand{\PHobjectif}[2]{\mbox{$#1\PHfrappeB^*\!#2$}}
\newcommand{\PHconcat}{::}
\newcommand{\PHneutralise}{\rtimes}

\def\PHget#1#2{{#1[#2]}}
%\newcommand{\PHchange}[2]{#1\langle #2 \rangle}
\newcommand{\PHchange}[2]{(#1 \Lleftarrow #2)}
\newcommand{\PHarcn}[2]{\mbox{$#1\PHneutralise#2$}}
\newcommand{\PHjoue}{\cdot}

\newcommand{\PHetat}[1]{\mbox{$\langle #1 \rangle$}}

% Notations spécifiques aux graphes d'états
\newcommand{\PHge}{\textcolor{red}{\mathcal{GE}}}
\newcommand{\PHt}{\mathcal{T}}
\newcommand{\GE}{\mathcal{GE}}
\newcommand{\GEt}{\mathcal{T}}
\newcommand{\GEl}{\PHl}
\newcommand{\GEa}{\PHa}
\newcommand{\GEva}[3]{#1 \stackrel{#2}{\longrightarrow} #3}
\newcommand{\GEval}[3]{#1 \stackrel{#2}{\Longrightarrow} #3}
\newcommand{\GEget}[2]{\PHget{#1}{#2}}



% Notations pour le modèle de Thomas (depuis thèse)
\def\levels{\mathsf{niveaux}}
\def\levelsA#1#2{\levels_+(#1\rightarrow #2)}
\def\levelsI#1#2{\levels_-(#1\rightarrow #2)}
\newcommand{\PHres}{\mathsf{Res}}

% Notations spécifiques au modèle de Thomas
\newcommand{\Kinconnu}{\emptyset}
\newcommand{\RRGva}[3]{#1 \stackrel{#2}{\longrightarrow} #3}
\newcommand{\RRGgi}{\mathcal{G}}
\newcommand{\RRGreg}[1]{\RRGgi_{#1}}
\newcommand{\RRGres}[2]{\PHres_{#1}(#2)}



% Notations spécifiques aux actions conjointes
\newcommand{\PHfrappeAconjointe}{\rightarrowtail}
\newcommand{\PHfrappec}[3]{\mbox{$#1\PHfrappeAconjointe#2\PHfrappeB#3$}}



% Notations spécifiques à ce document
\newcommand{\A}{}
\newcommand{\B}{'}
%\newcommand{\A}{_A}
%\newcommand{\B}{_B}
\newcommand{\simule}{\rho}
\newcommand{\relsimule}{\textcolor{red}{\lesssim}}
\newcommand{\abstraction}{\alpha}
\newcommand{\chap}[1]{\textcolor{blue}{\widehat{#1}}}

% Notations spécifiques à la traduction
\newcommand{\PHsc}{\ensuremath{sc}}
\newcommand{\PHsca}[1]{\ensuremath{\PHsc^{#1}}}
\newcommand{\PHfsc}{f_{sc}}
\newcommand{\PHjouables}{\mathsf{jouables}}


% Macros générales
\def\Pint{\textsc{PINT}}

% Notations générales pour PH
\newcommand{\PH}{\mathcal{PH}}
\newcommand{\PHs}{\Sigma}
\newcommand{\PHl}{L}
\newcommand{\PHp}{\textcolor{red}{\mathcal{P}}}
\newcommand{\PHproc}{\mathcal{P}}
\newcommand{\PHa}{\PHh}
\newcommand{\PHh}{\mathcal{H}}
\newcommand{\PHn}{\mathcal{N}}

\newcommand{\PHhitter}{\mathsf{hitter}}
\newcommand{\PHtarget}{\mathsf{target}}
\newcommand{\PHbounce}{\mathsf{bounce}}
\newcommand{\PHsort}{\Sigma}

\def\f#1{\mathsf{#1}}
\def\focals{\f{focals}}
\def\play{\cdot}
\def\configs#1{\mathbb C_{#1\rightarrow a}}

\newcommand{\PHfrappeA}{\rightarrow}
\newcommand{\PHfrappeB}{\Rsh}
\newcommand{\PHfrappe}[3]{#1\PHfrappeA#2\PHfrappeB#3}
\newcommand{\PHfrappebond}[2]{#1\PHfrappeB#2}
\newcommand{\PHobjectif}[2]{\mbox{$#1\PHfrappeB^*\!#2$}}
\newcommand{\PHconcat}{::}
\newcommand{\PHneutralise}{\rtimes}

\def\PHget#1#2{{#1[#2]}}
\newcommand{\PHchange}[2]{(#1 \Lleftarrow #2)}
\newcommand{\PHarcn}[2]{\mbox{$#1\PHneutralise#2$}}
\newcommand{\PHjoue}{\cdot}

\newcommand{\PHetat}[1]{\mbox{$\langle #1 \rangle$}}

% Notations spécifiques à ce papier
\newcommand{\PHdirectpredec}[1]{\PHs^{-1}(#1)}
\newcommand{\PHpredec}[1]{\f{pred}(#1)}
\newcommand{\PHpredecgene}[1]{\f{reg}({#1})}
\newcommand{\PHpredeccs}[1]{\PHpredec{#1} \setminus \Gamma}

\newcommand{\PHincl}[2]{#1 :: #2}

\def\ctx{\varsigma}
\def\ctxOverride{\Cap}
\def\PHstate#1{\langle #1 \rangle}

\def\DEF{\stackrel{\Delta}=}
\def\EQDEF{\stackrel{\Delta}\Leftrightarrow}

\def\intervalless{<_{[]}}

%\usepackage{ifthen}
\usepackage{tikz}
\usetikzlibrary{arrows,shapes}

\definecolor{lightgray}{rgb}{0.8,0.8,0.8}
\definecolor{lightgrey}{rgb}{0.8,0.8,0.8}

\tikzstyle{boxed ph}=[]
\tikzstyle{sort}=[fill=lightgray,rounded corners]
\tikzstyle{process}=[circle,draw,minimum size=15pt,fill=white,
font=\footnotesize,inner sep=1pt]
\tikzstyle{black process}=[process, fill=black,text=white, font=\bfseries]
\tikzstyle{gray process}=[process, draw=black, fill=lightgray]
\tikzstyle{current process}=[process, draw=black, fill=lightgray]
\tikzstyle{process box}=[white,draw=black,rounded corners]
\tikzstyle{tick label}=[font=\footnotesize]
\tikzstyle{tick}=[black,-]%,densely dotted]
\tikzstyle{hit}=[->,>=angle 45]
\tikzstyle{selfhit}=[min distance=30pt,curve to]
\tikzstyle{bounce}=[densely dotted,>=stealth',->]
\tikzstyle{hl}=[font=\bfseries,very thick]
\tikzstyle{hl2}=[hl]
\tikzstyle{nohl}=[font=\normalfont,thin]

\newcommand{\currentScope}{}
\newcommand{\currentSort}{}
\newcommand{\currentSortLabel}{}
\newcommand{\currentAlign}{}
\newcommand{\currentSize}{}

\newcounter{la}
\newcommand{\TSetSortLabel}[2]{
  \expandafter\repcommand\expandafter{\csname TUserSort@#1\endcsname}{#2}
}
\newcommand{\TSort}[4]{
  \renewcommand{\currentScope}{#1}
  \renewcommand{\currentSort}{#2}
  \renewcommand{\currentSize}{#3}
  \renewcommand{\currentAlign}{#4}
  \ifcsname TUserSort@\currentSort\endcsname
    \renewcommand{\currentSortLabel}{\csname TUserSort@\currentSort\endcsname}
  \else
    \renewcommand{\currentSortLabel}{\currentSort}
  \fi
  \begin{scope}[shift={\currentScope}]
  \ifthenelse{\equal{\currentAlign}{l}}{
    \filldraw[process box] (-0.5,-0.5) rectangle (0.5,\currentSize-0.5);
    \node[sort] at (-0.2,\currentSize-0.4) {\currentSortLabel};
   }{\ifthenelse{\equal{\currentAlign}{r}}{
     \filldraw[process box] (-0.5,-0.5) rectangle (0.5,\currentSize-0.5);
     \node[sort] at (0.2,\currentSize-0.4) {\currentSortLabel};
   }{
    \filldraw[process box] (-0.5,-0.5) rectangle (\currentSize-0.5,0.5);
    \ifthenelse{\equal{\currentAlign}{t}}{
      \node[sort,anchor=east] at (-0.3,0.2) {\currentSortLabel};
    }{
      \node[sort] at (-0.6,-0.2) {\currentSortLabel};
    }
   }}
  \setcounter{la}{\currentSize}
  \addtocounter{la}{-1}
  \foreach \i in {0,...,\value{la}} {
    \TProc{\i}
  }
  \end{scope}
}

\newcommand{\TTickProc}[2]{ % pos, label
  \ifthenelse{\equal{\currentAlign}{l}}{
    \draw[tick] (-0.6,#1) -- (-0.4,#1);
    \node[tick label, anchor=east] at (-0.55,#1) {#2};
   }{\ifthenelse{\equal{\currentAlign}{r}}{
    \draw[tick] (0.6,#1) -- (0.4,#1);
    \node[tick label, anchor=west] at (0.55,#1) {#2};
   }{
    \ifthenelse{\equal{\currentAlign}{t}}{
      \draw[tick] (#1,0.6) -- (#1,0.4);
      \node[tick label, anchor=south] at (#1,0.55) {#2};
    }{
      \draw[tick] (#1,-0.6) -- (#1,-0.4);
      \node[tick label, anchor=north] at (#1,-0.55) {#2};
    }
   }}
}
\newcommand{\TSetTick}[3]{
  \expandafter\repcommand\expandafter{\csname TUserTick@#1_#2\endcsname}{#3}
}

\newcommand{\myProc}[3]{
  \ifcsname TUserTick@\currentSort_#1\endcsname
    \TTickProc{#1}{\csname TUserTick@\currentSort_#1\endcsname}
  \else
    \TTickProc{#1}{#1}
  \fi
  \ifthenelse{\equal{\currentAlign}{l}\or\equal{\currentAlign}{r}}{
    \node[#2] (\currentSort_#1) at (0,#1) {#3};
  }{
    \node[#2] (\currentSort_#1) at (#1,0) {#3};
  }
}
\newcommand{\TSetProcStyle}[2]{
  \expandafter\repcommand\expandafter{\csname TUserProcStyle@#1\endcsname}{#2}
}
\newcommand{\TProc}[1]{
  \ifcsname TUserProcStyle@\currentSort_#1\endcsname
    \myProc{#1}{\csname TUserProcStyle@\currentSort_#1\endcsname}{}
  \else
    \myProc{#1}{process}{}
  \fi
}

\newcommand{\repcommand}[2]{
  \providecommand{#1}{#2}
  \renewcommand{#1}{#2}
}
\newcommand{\THit}[5]{
  \path[hit] (#1) edge[#2] (#3#4);
  \expandafter\repcommand\expandafter{\csname TBounce@#3@#5\endcsname}{#4}
}
\newcommand{\TBounce}[4]{
  (#1\csname TBounce@#1@#3\endcsname) edge[#2] (#3#4)
}

\newcommand{\TState}[1]{
  \foreach \proc in {#1} {
    \node[current process] (\proc) at (\proc.center) {};
  }
}


% Notations pour le modèle de Thomas (depuis thèse)
\newcommand{\GRN}{\mathcal{GRN}}
\newcommand{\IG}{\mathcal{G}}
\newcommand{\GRNreg}[1]{\Gamma^{-1}(#1)}
\newcommand{\GRNres}[2]{\mathsf{Res}_{#1}(#2)}
\newcommand{\GRNallres}[1]{\mathsf{Res}_{#1}}
\newcommand{\GRNget}[2]{\PHget{#1}{#2}}
\newcommand{\GRNetat}[1]{\PHetat{#1}}

\def\levels{\mathsf{levels}}
\def\levelsA#1#2{\levels_+(#1\rightarrow #2)}
\def\levelsI#1#2{\levels_-(#1\rightarrow #2)}

\newcommand{\Kinconnu}{\emptyset}
\newcommand{\RRGva}[3]{#1 \stackrel{#2}{\longrightarrow} #3}
\newcommand{\RRGgi}{\mathcal{G}}
\newcommand{\RRGreg}[1]{\RRGgi_{#1}}

\tikzstyle{grn}=[every node/.style={circle,draw=black,outer sep=2pt,minimum
                size=15pt,text=black}, node distance=1.5cm]
\tikzstyle{act}=[->,draw=black,thick,color=black]
\tikzstyle{inh}=[>=|,-|,draw=black,thick, text=black,label]
\tikzstyle{inf}=[->,draw=colorinf,thick,color=colorinf]
\tikzstyle{elabel}=[fill=none,text=black, above=-2pt,%sloped,
minimum size=10pt, outer sep=0, font=\scriptsize, draw=none]
\tikzstyle{sg}=[every node/.style={outer sep=2pt,minimum
                size=15pt,text=black}, node distance=2cm]

%\input{macros/macros-grn}

\usepackage{ifthen}
%\usepackage{tikz}
%\usetikzlibrary{arrows,shapes}

\definecolor{lightgray}{rgb}{0.8,0.8,0.8}
\definecolor{lightgrey}{rgb}{0.8,0.8,0.8}

\tikzstyle{boxed ph}=[]
\tikzstyle{sort}=[fill=lightgray,rounded corners]
\tikzstyle{process}=[circle,draw,minimum size=15pt,fill=white,
font=\footnotesize,inner sep=1pt]
\tikzstyle{black process}=[process, fill=black,text=white, font=\bfseries]
\tikzstyle{gray process}=[process, draw=black, fill=lightgray]
\tikzstyle{current process}=[process, draw=black, fill=lightgray]
\tikzstyle{process box}=[white,draw=black,rounded corners]
\tikzstyle{tick label}=[font=\footnotesize]
\tikzstyle{tick}=[black,-]%,densely dotted]
\tikzstyle{hit}=[->,>=angle 45]
\tikzstyle{selfhit}=[min distance=30pt,curve to]
\tikzstyle{bounce}=[densely dotted,->,>=latex]
\tikzstyle{hl}=[font=\bfseries,very thick]
\tikzstyle{hl2}=[hl]
\tikzstyle{nohl}=[font=\normalfont,thin]

\newcommand{\currentScope}{}
\newcommand{\currentSort}{}
\newcommand{\currentSortLabel}{}
\newcommand{\currentAlign}{}
\newcommand{\currentSize}{}

\newcounter{la}
\newcommand{\TSetSortLabel}[2]{
  \expandafter\repcommand\expandafter{\csname TUserSort@#1\endcsname}{#2}
}
\newcommand{\TSort}[4]{
  \renewcommand{\currentScope}{#1}
  \renewcommand{\currentSort}{#2}
  \renewcommand{\currentSize}{#3}
  \renewcommand{\currentAlign}{#4}
  \ifcsname TUserSort@\currentSort\endcsname
    \renewcommand{\currentSortLabel}{\csname TUserSort@\currentSort\endcsname}
  \else
    \renewcommand{\currentSortLabel}{\currentSort}
  \fi
  \begin{scope}[shift={\currentScope}]
  \ifthenelse{\equal{\currentAlign}{l}}{
    \filldraw[process box] (-0.5,-0.5) rectangle (0.5,\currentSize-0.5);
    \node[sort] at (-0.2,\currentSize-0.4) {\currentSortLabel};
   }{\ifthenelse{\equal{\currentAlign}{r}}{
     \filldraw[process box] (-0.5,-0.5) rectangle (0.5,\currentSize-0.5);
     \node[sort] at (0.2,\currentSize-0.4) {\currentSortLabel};
   }{
    \filldraw[process box] (-0.5,-0.5) rectangle (\currentSize-0.5,0.5);
    \ifthenelse{\equal{\currentAlign}{t}}{
      \node[sort,anchor=east] at (-0.3,0.2) {\currentSortLabel};
    }{
      \node[sort] at (-0.6,-0.2) {\currentSortLabel};
    }
   }}
  \setcounter{la}{\currentSize}
  \addtocounter{la}{-1}
  \foreach \i in {0,...,\value{la}} {
    \TProc{\i}
  }
  \end{scope}
}

\newcommand{\TTickProc}[2]{ % pos, label
  \ifthenelse{\equal{\currentAlign}{l}}{
    \draw[tick] (-0.6,#1) -- (-0.4,#1);
    \node[tick label, anchor=east] at (-0.55,#1) {#2};
   }{\ifthenelse{\equal{\currentAlign}{r}}{
    \draw[tick] (0.6,#1) -- (0.4,#1);
    \node[tick label, anchor=west] at (0.55,#1) {#2};
   }{
    \ifthenelse{\equal{\currentAlign}{t}}{
      \draw[tick] (#1,0.6) -- (#1,0.4);
      \node[tick label, anchor=south] at (#1,0.55) {#2};
    }{
      \draw[tick] (#1,-0.6) -- (#1,-0.4);
      \node[tick label, anchor=north] at (#1,-0.55) {#2};
    }
   }}
}
\newcommand{\TSetTick}[3]{
  \expandafter\repcommand\expandafter{\csname TUserTick@#1_#2\endcsname}{#3}
}

\newcommand{\myProc}[3]{
  \ifcsname TUserTick@\currentSort_#1\endcsname
    \TTickProc{#1}{\csname TUserTick@\currentSort_#1\endcsname}
  \else
    \TTickProc{#1}{#1}
  \fi
  \ifthenelse{\equal{\currentAlign}{l}\or\equal{\currentAlign}{r}}{
    \node[#2] (\currentSort_#1) at (0,#1) {#3};
  }{
    \node[#2] (\currentSort_#1) at (#1,0) {#3};
  }
}
\newcommand{\TSetProcStyle}[2]{
  \expandafter\repcommand\expandafter{\csname TUserProcStyle@#1\endcsname}{#2}
}
\newcommand{\TProc}[1]{
  \ifcsname TUserProcStyle@\currentSort_#1\endcsname
    \myProc{#1}{\csname TUserProcStyle@\currentSort_#1\endcsname}{}
  \else
    \myProc{#1}{process}{}
  \fi
}

\newcommand{\repcommand}[2]{
  \providecommand{#1}{#2}
  \renewcommand{#1}{#2}
}
\newcommand{\THit}[5]{
  \path[hit] (#1) edge[#2] (#3#4);
  \expandafter\repcommand\expandafter{\csname TBounce@#3@#5\endcsname}{#4}
}
\newcommand{\TBounce}[4]{
  (#1\csname TBounce@#1@#3\endcsname) edge[#2] (#3#4)
}

\newcommand{\TState}[1]{
  \foreach \proc in {#1} {
    \node[current process] (\proc) at (\proc.center) {};
  }
}



\hyphenpenalty 1000

\newcommand{\FIG}[4]
{\begin{figure}[!hbt]
\begin{center}
\rotatebox{270}{\includegraphics[width=#1]{#2}}
\caption{\label{fig:#3}#4\vspace{-5mm}}
\end{center}
\end{figure}}



\begin{document}

\titre{Integration des données de séries temporelles à un model formel}
\auteur{Louis Fippo Fitime$^1$, Andrea Beica$^1$, Olivier Roux$^1$ and Carito Guziolowski$^{1}$}
\affiliation{$^1 $
LUNAM Universit\'e, \'Ecole Centrale de Nantes, IRCCyN UMR CNRS 6597\\
(Institut de Recherche en Communications et Cybern\'etique de Nantes)\\
1 rue de la No\"e -- B.P. 92101 -- 44321 Nantes Cedex 3, France.\\
\texttt{Louis.Fippo-Fitime@irccyn.ec-nantes.fr}}
\\

\section*{Abstract}
\vspace{-2mm}

L'étude et la compréhension des sytèmes biologiques repose sur notre capacité à construire des abstractions de ces systèmes 
afin d'en étudier les propriétés. De nombreux formalismes ont été proposé pour atteindre cet objectif.  Ces formalismes n'ont 
pas tous les mêmes fondements théoriques et les mêmes puissances d'analyse. En général, nous choisirons un formalisme en fonction 
d'un aspect bien précis du systeme que nous voulons étudier. 

Dans le cadre de notre travail nous avons proposé une étude du processus de différenciation des cellules de la peau grace à une 
modélisation de son sytème  biologique à l'aide du formalisme des frappes de processus. L'idée de base était de pouvoir intégrer
dans un modèle hybride les données temporelles de certains composants du réseau receuillis après une expérience réelle.  
Nous avons donc proposé une démarche méthodologique rigoureuse bsasée sur la description du système biologique et les résultats 
des expérimentations.

\vspace{-2mm}



\section{Introduction}

L'objet de ce travail est la modélisation, la simulation et l'analyse des systèmes biologiques en se servant des frappes de processus. 

Le but est de parcourir le travail éffectué dans la première partie des travaux de 
recherche qui visent à introduire les données de séries temporelles dans les modèles formels d'études 
des systèmes complexes. Nous nous intéressons ici à l'étude d'un grand(~$200$ noeuds et ~$400$ interactions) réseau biologique. 
Dans la section suivante, nous allons présenter les données qui nous ont servi de base pour cette modélisation. Nous y présenterons 
le graphe des interactions qui décrit le réseau biologique et les données de séries temporelles qui ont permi d'estimer les paramètres temporelles. 
Dans la section suivante, nous présenterons le modèle des frappes de processsus. Cette présentation consistera a donner une défition formelle du 
formalisme et présenter quelques propriétés. Par la suite nous présenterons les résultats obtenus dans le cadre de ce travail. 

\vspace{3mm}
\noindent

\section{Les données}

Dans cette première section nous présentons les données qui nous ont servi de support pour la modélisation de notre système. 

Nous allons nous focaliser sur deux éléments principaux: le graphe des interaction qui décrit les composants du sytème et 
les les différentes interactions qui existent entre ces composants, les données d'expressions des gènes. 


\subsection{Le graphe des interactions}
Les graphe d'interactions qui décrit le processus de cicatrisation de la peau est un réseau de signalisation qui décrit 
les composants et les interactions entre les différents composants. C'est un large réseau qui compte près de $200$ composants 
et à peu près $400$ interactions entre ces composants. 
Parmis les composants nous avons les gènes, les protéines, les complexes et les indicateurs d'états cellulaires.
Les interactions entre les composants sont de plusieurs natures. Nous avons principalement les activations entre composants, les inibitions,
les décompositions de complexes. 


\subsection{Les données de séries temporelles}

Les données de séries temporelles sont les données obtenues après une expérience réelle en laboratoire. 

\section{Le framework de modélisation: le formalisme des frappes de processus}

Dans cette section nous allons définir et donner la sémentaique du formalisme des frappes de processus. Le formalisme des frappes des processus 
regroupe un ensemble de processus(généralement concurrents) en un ensemble fini de sortes. Un processus est associé à une et une seule sorte et 
est dénoté par $a_{i}$ où $a$ le nom de la sorte et $i$ est l'identificateur du processus dans la sorte a. A chaque instant un et un seul processus 
de chaque sorte est présent. L'ensemble des processus actif à un moment donné représentent l'état du système.

Les processus interagissent entre eux à travers les actions concurrentes regroupé dans un ensemble d'actions. Une action décrit le changement d'un 
processus d'une sorte par un autre processus de la même sorte.	

\subsection{Défintion formelle des frappes de processus}

\subsection{Propriétés des frappes de processus}

% construction des Frappes de processus par d'autres  frappes de processus(union, intersection, ...)

\section{Les contraintes de la comparaison}

% la nécessité de discretiser les données 

% présentation de l'algorithme de discretisation 

% présentation de la notion de trace et équivalence de trace.

\section{Résultats}


\subsection{Du RSTC network vers PH model}

%introduction des patterns

Pour traduire notre réseau de signalisation en frappe de processus, nous avons identifié un ensemble de patterns dans le réseau de signalisation auxquels nous avons 
associé les patterns équivalents en frappe de processus. Le tableau suivant répertorie les principaux patterns, leurs équivalents en frappes de processus et une brève 
description de leur sémantique. 

% construction des Frappes de processus par d'autres  frappes de processus(union, intersection, ...)

\subsection{intégration des données de séries temporelles}

%Estimation des paramètres temporelles (t1 t2 t3 t4 t5 ...)
L'idée de l'estimation des paramètres temporelles est de déterminer les différents instant de temps pour les quels on considère un changement 
d'état à partir des niveaux d'expressions. La détermination des changements d'état va donc s'appuyer sur un choix au préalable des seuils qui 
vont déterminer les différents niveaux d'activités. Ainsi dès que le niveau d'expression va traverser un seuil, on va considérer qu'il y a un 
changement d'état au niveau du composant. Ce qui doit donc se traduire par une action dans notre modèle qui va entrainer cette dynamique.

Dans un premier temps, nous devons donc déterminer les différents instant $t_{i}$ correspondant à un changement d'état. Par la suite nous devons 
donner les rates aux actions responsables de cette dynamique de façon à pouvoir les observer aux instants estimés.

%intégration des paramètres temporelles sous forme de rate et d'absorption de stochasticité ( r et sa)

Le rate d'une action peut être vu comme une approximation de l'inverse  du temps moyen nécessaire pour l'exécution de cette action. L'idée 
c'est de faire en sorte que le temp moyen d'une action corresponde à l'interval de temp qui sépare cette action de la précédente.



\subsection{Simulation et analyse des résultats}

% présenter les différents cas de figure des simulations 
Dans cette partie nous présentons les résultats des simulations réalisées. Nous allons présnter différents cas de figures qui correspondent 
aux choix de simulatios qui nous ont permi de comprendre l'évolution de notre système et d'améliorer la qualité de nos simulations. Pour chacune de ces 
simulations nous avons choisi les paramètres temporelles et stochastiques de la façon suivante:
\begin{enumerate}
 \item Nous avons choisi une absortion de stochasticité de $50$ pour toutes les actions: l'objectif de ce choix est de raffiner au mieux l'interval de tir.
 \item Nous avons affecté les rates estimés aux actions qui influencent les gènes dont nous avons pu mesurer le niveau d'expression.
 \item Nous avons affecté une valeur de $10$ comme rate pour les actions qui influencent les gènes dont nous n'avons pas d'information sur les niveaux d'expressions. 
\end{enumerate}


% premier cas: Le graphe brute 
\subsubsection{Le graphe brute}
Les résultats suivants correspondent aux simulations réealisées sur le graphe complet de notre système après insertion des paramètres temporelles estimés 
et du facteur d'absortion de stochasticité. 


%deuxième cas: décomposition en sous graphes qui ne contiennent pas de boucles et d'auto dégradation de composants
\subsubsection{Décomposition en sous graphe}
Afin de simplifier l'analyse et la comparaison des traces d'exécutions, nous avons proposé une analyse par sous-graphe du réseau représentant notre système 
biologique. 

%expliquer la construction des sous graphes




% troisième cas: pas de feedback et nous introduisons une auto dégradation des facteurs de transcription

%quatrième cas de figure: nous introduisons des feedbacks


\subsubsection{Intégration des sous graphes}
%intérêtes des sortes de synchronisations




\section{Conclusions}
% A faire après la réunion

\subsection{Perspectives}

\begin{enumerate}
 \item Sortes de synchronisation???
 \item Extension de la définition(horloges explicites, relachement des contraintes)
\end{enumerate}


\bibliographystyle{plain}
\bibliography{biblio.bib}


\appendix

\clearpage
\appendix


\end{document}
