\documentclass[11pt,a4paper,twoside]{epig}
\input{epig-macros}

\hyphenpenalty 1000

\newcommand{\FIG}[4]
{\begin{figure}[!hbt]
\begin{center}
\rotatebox{270}{\includegraphics[width=#1]{#2}}
\caption{\label{fig:#3}#4\vspace{-5mm}}
\end{center}
\end{figure}}



\begin{document}

\titre{Integrating time-series data on large-scale cell-based models: application to skin differentiation}
\auteur{Louis Fippo Fitime$^1$, Andrea Beica$^1$, Olivier Roux$^1$ and Carito Guziolowski$^{1}$}
\affiliation{$^1 $
LUNAM Universit\'e, \'Ecole Centrale de Nantes, IRCCyN UMR CNRS 6597\\
(Institut de Recherche en Communications et Cybern\'etique de Nantes)\\
1 rue de la No\"e -- B.P. 92101 -- 44321 Nantes Cedex 3, France.\\
\texttt{Louis.Fippo-Fitime@irccyn.ec-nantes.fr}}
\\

\section*{Abstract}
\vspace{-2mm}
The way living organisms work and develop themselves is controlled by large and complex 
networks of genes, proteins, small molecules, and their interactions, so-called biological 
regulatory networks. Confronting time-series gene expression data with models may allow us to
examine and characterize the dynamics of elements that compose such a regulatory network. 
In this work, we propose a way to model and simulate large-scale regulatory networks, by using the 
Process Hitting (PH) framework, in order to verify if the model describes the experimental measures.
The preliminary work presented here proposes: (1) a semi-automatic method to build a PH from
a regulatory network of biochemical reactions, (2) a discretization scheme of the continuous time-series measurements, 
and (3) an approach to estimate the PH stochastic simulation parameters in an unbiased manner.
\vspace{-2mm}

\end{document}
